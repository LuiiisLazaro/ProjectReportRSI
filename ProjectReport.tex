\documentclass{report}
\usepackage[spanish]{babel}
\usepackage{natbib}
\usepackage{url}
\usepackage[utf8]{inputenc}
\usepackage{amsmath}
\usepackage{graphicx}
\graphicspath{{images/}}
\usepackage{parskip}
\usepackage{fancyhdr}
\usepackage{vmargin}
\usepackage{subfig}
\usepackage[hidelinks]{hyperref}
\usepackage{enumerate}
\usepackage[usenames]{color}
\usepackage{soul}
\usepackage{float}
\usepackage[T1]{fontenc}
\usepackage{verbatim}
\usepackage{multicol}
\usepackage{listings}
\usepackage{multirow}
\usepackage{booktabs}

\setmarginsrb{3 cm}{2.5 cm}{3 cm}{2.5 cm}{1 cm}{1.5 cm}{1 cm}{1.5 cm}

\title{\textsc{Reporte Proyecto Requerimientos de Proceso de Viáticos}}
\author{Luis Angel Hernández Lázaro \\ Daniel Mendez Cruz}
\date{\today}

\makeatletter
\let\thetitle\@title
\let\theauthor\@author
\let\thedate\@date
\makeatother

\pagestyle{fancy}
\fancyhf{}
\rhead{\theauthor}
\lhead{\thetitle}
\cfoot{\thepage}
\setcounter{secnumdepth}{5}

\begin{document}

%%%%%%%%%%%%%%%%%%%%%%%%%%%%%%%%%%%%%%%%%%%%%%%%%%%%%%%%%%%%%%%%%%%%%%%%%%%%%%%%%%%%%%%%%

\begin{titlepage}
	\centering
    \vspace*{0.5 cm}
    
    
    \begin{figure}
		\centering
		\subfloat{
			\label{logoCimat}
		    \includegraphics[scale = 0.15]{images/0cover/logoCimat.png}
		}
	\end{figure}
    \textsc{\LARGE Centro de Investigación en Matemáticas A.C.}\\[1.0 cm]
	\textsc{\Large Maestría en Ingeniería de Software}\\[0.5 cm]
	\textsc{\large Requerimientos de Ingeniería de Software\\Dr. Hugo Arnoldo Mitre}\\[0.5 cm]
	\rule{\linewidth}{0.2 mm} \\[0.4 cm]
	{ \huge \bfseries \thetitle}\\ \textsc{\large Daniel Mendez Cruz\\ Luis Angel Hernández Lázaro}
	\rule{\linewidth}{0.2 mm} \\[1.5 cm]
	
	\begin{minipage}{0.5\textwidth}
		\begin{flushleft} \large
			\emph{Correo:}\\
			daniel.mendez@cimat.mx
		\end{flushleft}
	\end{minipage}~
	\begin{minipage}{0.4\textwidth}
		\begin{flushright} \large
			\emph{Correo:} \\
			luis.hernandez@cimat.mx
		\end{flushright}
	\end{minipage}\\[2 cm]
		
	{\large \thedate}\\[2 cm]
 
	\vfill
	
\end{titlepage}

%%%%%%%%%%%%%%%%%%%%%%%%%%%%%%%%%%%%%%%%%%%%%%%%%%%%%%%%%%%%%%%%%%%%%%%%%%%%%%%%%%%%%%%%%

\tableofcontents
\pagebreak

%%%%%%%%%%%%%%%%%%%%%%%%%%%%%%%%%%%%%%%%%%%%%%%%%%%%%%%%%%%%%%%%%%%%%%%%%%%%%%%%%%%%%%%%%
\chapter{Identificación del Problema}
    
    \section{Introducción}

En el problema que se abordó sobre el casos de uso, fue sobre las actividades que realizá Jessica Garrido Sánchez quien se encarga de administrar 

\chapter{Análisis del Problema}

    \section{Introducción}
    En esta sección del trabajo se encuentran descritos los artefactos creados para visualizar el análisis de los requerimientos,entre los artefactos que se han creado se encuentran: \emph{Historias de Usuario}, \emph{Diagrama de Casos de Uso}, \emph{Diagrama de BPMN (Business Process Model and Notacion)} y el \emph{Diagrama de Clases}.    
    
    \section{Historias de Usuario}
    
    \begin{description}
        \item[<Etiqueta>]: HU1
        \begin{description}
            \item[``Yo como]: <<rol>>, Usuario
            \item[quiero]: <<Objetivo/Deseo>> dinero
            \item[para]: <<Beneficio>> ``gastar''.
        \end{description}
        \item[Fecha]: 8-Dic-15
        \item[Responsable]: Daniel Mendez Cruz
    \end{description}
    
<Label> :
"As a <role>, I
want <goal/desire> so
that <benefit> “.
<date>
<responsible>.    
    
    \begin{table}
        \begin{center}
            \caption{H}
            \label{table:ResultsSearchStringOutStandingStudies}
            \begin{tabular}{| p{6cm} | p{1cm} | p{1cm} | p{1cm} | p{4cm} |}
                \toprule
                \hline
                \textbf{Fuente /Cadena} & \textbf{ss1} & \textbf{ss2} & \textbf{ss3} & \textbf{Total por Biblioteca Digital}\\ \hline
                ACM Digital Library & 13 & 4 & 3 & 20 \\ \hline
            \end{tabular}
        \end{center}
    \end{table}
    
    \section{Diagrama de Casos de Uso}
    
    \begin{figure}
		\centering
		\subfloat{
			\label{logoCimat}
		    \includegraphics[scale = 0.80]{images/1models/useCase.png}
		}
	\end{figure}    
    
    
    \section{Diagrama BPMN (Business Process Model and Notacion)}
    
    \section{Diagrama de clases}
    
    \section{Conclusiones}
       
\chapter{Lecciones Aprendidas de una Identificación de Requerimientos}
    
    \section{Lecciones Aprendidas}
    
    \section{Trabajos Futuros}
    
    \section{conclusiones}

\end{document}
